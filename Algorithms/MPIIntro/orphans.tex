Stuff that needs a home:

    There are three main differences to keep in mind between mpi4py and MPI in C:
    \begin{itemize}
        \item Python is array-based. C and Fortran are not.
        \item mpi4py is object oriented. MPI in C is not.
            Because of this, there are some minor changes between the mpi4py implementation of 
            MPI and the official MPI specification.

            For instance, the MPI Communicator in mpi4py is a Python class and MPI functions 
            like \li{Get_size()} or \li{Get_rank()} are instance methods of the communicator class. 
            Throughout these MPI labs, you will see functions like \li{Get_rank()} presented as 
            \li{Comm.Get_rank()} where it is implied that \li{Comm} is a communicator object.
        \item mpi4py supports two methods of communication to implement each of the basic MPI commands. 
            They are the upper and lower case commands (e.g. \li{Bcast(...)} and \li{bcast(...)}). 
            The uppercase implementations use traditional MPI datatypes while the lower case use 
            Python's pickling method. Pickling offers extra convenience to using mpi4py, 
            but the traditional method is faster. In these labs, we will only use the uppercase functions.
    \end{itemize}

Do we even need this??? Maybe as a recap at enc of the old lab 1
    Here is the syntax for \li{Get_size()} and \li{Get_rank()}, where \li{Comm} is a communicator object:
    \begin{description}
    \item[Comm.Get\_size()]
    Returns an integer representing the number of processes in the communicator.
    \emph{This method will return the same number to every process.}
    Example:
    % TODO add output?
    \begin{lstlisting}
    from mpi4py import MPI
    SIZE = MPI.COMM_WORLD.Get_size()
    print "The number of processes is {SIZE}.".format(**locals())
    \end{lstlisting}
    \item[Comm.Get\_rank()]
    Returns an integer representing the rank of the calling process in the communicator.
    \emph{This method will return a different value to each process.}
    Example:
    % TODO add output?
    \begin{lstlisting}
    from mpi4py import MPI
    RANK = MPI.COMM_WORLD.Get_rank()
    print "My rank is {RANK}.".format(**locals())
    \end{lstlisting}
    \end{description}

tell what root node means
