\lab{Python}{F2PY}{F2PY}
\label{lab:f2py}
\objective{Learn how to use F2PY, especially for wrapping Fortran code involving arrays.}

Here we will explore another simple way to access external code from Python.
Various utilities exist to do this, and many are specialized for specific tasks.
F2PY is a utility that can be used to automatically wrap Fortran subroutines so that they are callable from Python.
It is included with NumPy and in most cases does not usually require heavy configuration for use.
The primary requirement for using F2PY is a working Fortran compiler.
If you have access to GFortran, the Fortran equivalent of gcc, it should work well.

\section*{Examples}

F2PY is very simple to use.
Many simple Fortran files can be compiled \emph{automatically} without any hand-written interfaces between languages.
Windows 64 bit users, if you encounter errors, see the Windows 64 bit install guide.
We will start by wrapping the Fortran function we wrapped as a part of Lab \ref{lab:cythonwrap}.
Here we will wrap the subroutine in unmodified form.
In Lab \ref{lab:cythonwrap} we modified the Fortran file to make it callable from C.
F2PY will take care of this for us so that we can wrap the subroutine directly.
Here is the file in its unmodified form:
\lstinputlisting[style=fromfile, language=Fortran]{ftridiag.f90}
We can wrap this file by running the following in our terminal:
\begin{lstlisting}[style=ShellInput]
python f2py.py -c ftridiag.f90 -m ftridiag
\end{lstlisting}
We can also tell F2PY to pass an optimization flag to the compiler using the \li{--opt} flag, as in:
\begin{lstlisting}[style=ShellInput]
python f2py.py -c ftridiag.f90 -m ftridiag --opt="-O3"
\end{lstlisting}
Depending on your system configuration, you may be able to call F2PY without expressly calling python, or without including its file extension (or both).

Running these commands will create a file called \li{ftridiag.pyd} on a Windows computer or \li{ftridiag.so} on a Linux or Mac based computer.
This file is a Python module that is linked against the compiled Fortran code.
All the details about array type, etc. have been infered by F2PY.
We can test the module the same way we tested the module we built using Cython.
\lstinputlisting[style=fromfile]{ftridiag_test.py}
For our example here we compiled the subroutine by itself instead of compiling an entire module.
If the function were a part of a Fortran module \li{tridiag} F2PY would have made it available to us as \li{ftridiag.tridiag.ftridiag}.

% Show example of wrapping FORTRAN using Cython.

% Introduce f2py, teach how to wrap FORTRAN automatically.

% Wrap a C function using f2py to teach about signature files.

% Wrap a LAPACK function (dgtsv?), have them do it too.

% Discuss how memory layout affects performance.

% Discuss programmer time value.